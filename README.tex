% Options for packages loaded elsewhere
\PassOptionsToPackage{unicode}{hyperref}
\PassOptionsToPackage{hyphens}{url}
%
\documentclass[
]{article}
\usepackage{amsmath,amssymb}
\usepackage{lmodern}
\usepackage{iftex}
\ifPDFTeX
  \usepackage[T1]{fontenc}
  \usepackage[utf8]{inputenc}
  \usepackage{textcomp} % provide euro and other symbols
\else % if luatex or xetex
  \usepackage{unicode-math}
  \defaultfontfeatures{Scale=MatchLowercase}
  \defaultfontfeatures[\rmfamily]{Ligatures=TeX,Scale=1}
\fi
% Use upquote if available, for straight quotes in verbatim environments
\IfFileExists{upquote.sty}{\usepackage{upquote}}{}
\IfFileExists{microtype.sty}{% use microtype if available
  \usepackage[]{microtype}
  \UseMicrotypeSet[protrusion]{basicmath} % disable protrusion for tt fonts
}{}
\makeatletter
\@ifundefined{KOMAClassName}{% if non-KOMA class
  \IfFileExists{parskip.sty}{%
    \usepackage{parskip}
  }{% else
    \setlength{\parindent}{0pt}
    \setlength{\parskip}{6pt plus 2pt minus 1pt}}
}{% if KOMA class
  \KOMAoptions{parskip=half}}
\makeatother
\usepackage{xcolor}
\IfFileExists{xurl.sty}{\usepackage{xurl}}{} % add URL line breaks if available
\IfFileExists{bookmark.sty}{\usepackage{bookmark}}{\usepackage{hyperref}}
\hypersetup{
  hidelinks,
  pdfcreator={LaTeX via pandoc}}
\urlstyle{same} % disable monospaced font for URLs
\usepackage[margin=1in]{geometry}
\usepackage{color}
\usepackage{fancyvrb}
\newcommand{\VerbBar}{|}
\newcommand{\VERB}{\Verb[commandchars=\\\{\}]}
\DefineVerbatimEnvironment{Highlighting}{Verbatim}{commandchars=\\\{\}}
% Add ',fontsize=\small' for more characters per line
\usepackage{framed}
\definecolor{shadecolor}{RGB}{248,248,248}
\newenvironment{Shaded}{\begin{snugshade}}{\end{snugshade}}
\newcommand{\AlertTok}[1]{\textcolor[rgb]{0.94,0.16,0.16}{#1}}
\newcommand{\AnnotationTok}[1]{\textcolor[rgb]{0.56,0.35,0.01}{\textbf{\textit{#1}}}}
\newcommand{\AttributeTok}[1]{\textcolor[rgb]{0.77,0.63,0.00}{#1}}
\newcommand{\BaseNTok}[1]{\textcolor[rgb]{0.00,0.00,0.81}{#1}}
\newcommand{\BuiltInTok}[1]{#1}
\newcommand{\CharTok}[1]{\textcolor[rgb]{0.31,0.60,0.02}{#1}}
\newcommand{\CommentTok}[1]{\textcolor[rgb]{0.56,0.35,0.01}{\textit{#1}}}
\newcommand{\CommentVarTok}[1]{\textcolor[rgb]{0.56,0.35,0.01}{\textbf{\textit{#1}}}}
\newcommand{\ConstantTok}[1]{\textcolor[rgb]{0.00,0.00,0.00}{#1}}
\newcommand{\ControlFlowTok}[1]{\textcolor[rgb]{0.13,0.29,0.53}{\textbf{#1}}}
\newcommand{\DataTypeTok}[1]{\textcolor[rgb]{0.13,0.29,0.53}{#1}}
\newcommand{\DecValTok}[1]{\textcolor[rgb]{0.00,0.00,0.81}{#1}}
\newcommand{\DocumentationTok}[1]{\textcolor[rgb]{0.56,0.35,0.01}{\textbf{\textit{#1}}}}
\newcommand{\ErrorTok}[1]{\textcolor[rgb]{0.64,0.00,0.00}{\textbf{#1}}}
\newcommand{\ExtensionTok}[1]{#1}
\newcommand{\FloatTok}[1]{\textcolor[rgb]{0.00,0.00,0.81}{#1}}
\newcommand{\FunctionTok}[1]{\textcolor[rgb]{0.00,0.00,0.00}{#1}}
\newcommand{\ImportTok}[1]{#1}
\newcommand{\InformationTok}[1]{\textcolor[rgb]{0.56,0.35,0.01}{\textbf{\textit{#1}}}}
\newcommand{\KeywordTok}[1]{\textcolor[rgb]{0.13,0.29,0.53}{\textbf{#1}}}
\newcommand{\NormalTok}[1]{#1}
\newcommand{\OperatorTok}[1]{\textcolor[rgb]{0.81,0.36,0.00}{\textbf{#1}}}
\newcommand{\OtherTok}[1]{\textcolor[rgb]{0.56,0.35,0.01}{#1}}
\newcommand{\PreprocessorTok}[1]{\textcolor[rgb]{0.56,0.35,0.01}{\textit{#1}}}
\newcommand{\RegionMarkerTok}[1]{#1}
\newcommand{\SpecialCharTok}[1]{\textcolor[rgb]{0.00,0.00,0.00}{#1}}
\newcommand{\SpecialStringTok}[1]{\textcolor[rgb]{0.31,0.60,0.02}{#1}}
\newcommand{\StringTok}[1]{\textcolor[rgb]{0.31,0.60,0.02}{#1}}
\newcommand{\VariableTok}[1]{\textcolor[rgb]{0.00,0.00,0.00}{#1}}
\newcommand{\VerbatimStringTok}[1]{\textcolor[rgb]{0.31,0.60,0.02}{#1}}
\newcommand{\WarningTok}[1]{\textcolor[rgb]{0.56,0.35,0.01}{\textbf{\textit{#1}}}}
\usepackage{graphicx}
\makeatletter
\def\maxwidth{\ifdim\Gin@nat@width>\linewidth\linewidth\else\Gin@nat@width\fi}
\def\maxheight{\ifdim\Gin@nat@height>\textheight\textheight\else\Gin@nat@height\fi}
\makeatother
% Scale images if necessary, so that they will not overflow the page
% margins by default, and it is still possible to overwrite the defaults
% using explicit options in \includegraphics[width, height, ...]{}
\setkeys{Gin}{width=\maxwidth,height=\maxheight,keepaspectratio}
% Set default figure placement to htbp
\makeatletter
\def\fps@figure{htbp}
\makeatother
\setlength{\emergencystretch}{3em} % prevent overfull lines
\providecommand{\tightlist}{%
  \setlength{\itemsep}{0pt}\setlength{\parskip}{0pt}}
\setcounter{secnumdepth}{-\maxdimen} % remove section numbering
\ifLuaTeX
  \usepackage{selnolig}  % disable illegal ligatures
\fi

\author{}
\date{\vspace{-2.5em}}

\begin{document}

\hypertarget{section}{%
\section{}\label{section}}

\hypertarget{indicadores-oasisbr}{%
\subsection{Indicadores oasisbr}\label{indicadores-oasisbr}}

\hypertarget{documentauxe7uxe3o-r-shiny-app}{%
\subsubsection{Documentação R-Shiny
app}\label{documentauxe7uxe3o-r-shiny-app}}

\hypertarget{indicadores-oasisbr---documentauxe7uxe3o-r-shiny-app}{%
\subsection{Indicadores oasisbr - Documentação R-Shiny
App}\label{indicadores-oasisbr---documentauxe7uxe3o-r-shiny-app}}

\begin{enumerate}
\def\labelenumi{\arabic{enumi}.}
\tightlist
\item
  Acesso à API e download de arquivo com indicadores
\item
  Análise exploratória dos dados
\item
  Instalação de servidores RStudio e RShiny
\item
  Aplicação R-Shiny
\end{enumerate}

\hypertarget{download-de-arquivos-com-informauxe7uxf5es-sobre-os-indicadores}{%
\subsection{Download de arquivos com informações sobre os
indicadores}\label{download-de-arquivos-com-informauxe7uxf5es-sobre-os-indicadores}}

\begin{enumerate}
\def\labelenumi{\arabic{enumi}.}
\tightlist
\item
  Os indicadores da oasisbr são disponibilizados via api.
\end{enumerate}

\begin{Shaded}
\begin{Highlighting}[]
\NormalTok{oasisbrAPILink }\OtherTok{\textless{}{-}} \StringTok{"https://oasisbr.ibict.br/vufind/api/v1/search?\&type=AllFields\&page=0\&limit=0\&sort=relevance\&facet[]=author\_facet\&facet[]=dc.subject.por.fl\_str\_mv\&facet[]=eu\_rights\_str\_mv\&facet[]=dc.publisher.program.fl\_str\_mv\&facet[]=dc.subject.cnpq.fl\_str\_mv\&facet[]=publishDate\&facet[]=language\&facet[]=format\&facet[]=institution\&facet[]=dc.contributor.advisor1.fl\_str\_mv"}
\end{Highlighting}
\end{Shaded}

\begin{center}\rule{0.5\linewidth}{0.5pt}\end{center}

\hypertarget{indicadores-de-evoluuxe7uxe3o}{%
\subsection{Indicadores de
evolução:}\label{indicadores-de-evoluuxe7uxe3o}}

\url{https://api-oasisbr.ibict.br/api/v1/evolution-indicators?init=10/10/2017\&end=10/10/2021}

\begin{center}\rule{0.5\linewidth}{0.5pt}\end{center}

\hypertarget{documentauxe7uxe3o-indicadores-de-evoluuxe7uxe3o}{%
\subsection{Documentação indicadores de
evolução:}\label{documentauxe7uxe3o-indicadores-de-evoluuxe7uxe3o}}

\url{https://api-oasisbr.ibict.br/api/v1/doc/\#/default/EvolutionIndicatorsController_find}

\begin{center}\rule{0.5\linewidth}{0.5pt}\end{center}

Mais informações sobre a api:\\
\url{https://oasisbr.ibict.br/vufind/api/v1/}\strut \\
Mais informações sobre \texttt{facets}:\\
\href{\%60https://solr.apache.org/guide/6_6/faceting.html\%60}{Apache
Solr Reference Guide 6.6 - Faceting}

É feito o download do arquivo em formato \texttt{JSON} via pacote
\texttt{jsolinte}, utilizando-se a função \texttt{fromJSON}.

\begin{Shaded}
\begin{Highlighting}[]
\FunctionTok{library}\NormalTok{(jsonlite)}
\NormalTok{oasisbrDF }\OtherTok{\textless{}{-}} \FunctionTok{fromJSON}\NormalTok{(oasisbrAPILink)}
\end{Highlighting}
\end{Shaded}

\begin{center}\rule{0.5\linewidth}{0.5pt}\end{center}

O arquivo possui a seguinte estrutura:

\begin{Shaded}
\begin{Highlighting}[]
\FunctionTok{names}\NormalTok{(oasisbrDF)}
\end{Highlighting}
\end{Shaded}

\begin{verbatim}
## [1] "resultCount" "facets"      "status"
\end{verbatim}

\begin{center}\rule{0.5\linewidth}{0.5pt}\end{center}

Dentro da lista \texttt{resultCount}, encontra-se \textbf{quantidade de
documentos} recuperados:

\begin{Shaded}
\begin{Highlighting}[]
\NormalTok{oasisbrDF}\SpecialCharTok{$}\NormalTok{resultCount}
\end{Highlighting}
\end{Shaded}

\begin{verbatim}
## [1] 3010384
\end{verbatim}

\begin{center}\rule{0.5\linewidth}{0.5pt}\end{center}

Dentro da lista \texttt{facets}, encontram-se \textbf{dez listas para
dez variáveis diferentes}:

\begin{Shaded}
\begin{Highlighting}[]
\FunctionTok{names}\NormalTok{(oasisbrDF}\SpecialCharTok{$}\NormalTok{facets)}
\end{Highlighting}
\end{Shaded}

\begin{verbatim}
##  [1] "author_facet"                      "dc.subject.por.fl_str_mv"         
##  [3] "eu_rights_str_mv"                  "dc.publisher.program.fl_str_mv"   
##  [5] "dc.subject.cnpq.fl_str_mv"         "publishDate"                      
##  [7] "language"                          "format"                           
##  [9] "institution"                       "dc.contributor.advisor1.fl_str_mv"
\end{verbatim}

As informações sobre as variáveis se encontram no documento
\texttt{Padrão\ de\ Metadados\ da\ BDTD\ -\ MTD3-BR\ v.2017.pdf}

Todas as listas possuem \textbf{4 colunas}:

\begin{Shaded}
\begin{Highlighting}[]
\CommentTok{\#Exemplo: author\_facet}
\FunctionTok{names}\NormalTok{(oasisbrDF}\SpecialCharTok{$}\NormalTok{facets}\SpecialCharTok{$}\NormalTok{author\_facet)}
\end{Highlighting}
\end{Shaded}

\begin{verbatim}
## [1] "value"      "translated" "count"      "href"
\end{verbatim}

A coluna \texttt{value} representa o valor, \texttt{translated} o valor
traduzido, \texttt{count} a frequência e \texttt{href} o hyperlink.

\begin{Shaded}
\begin{Highlighting}[]
\FunctionTok{head}\NormalTok{(oasisbrDF}\SpecialCharTok{$}\NormalTok{facets}\SpecialCharTok{$}\NormalTok{author\_facet)}
\end{Highlighting}
\end{Shaded}

\begin{verbatim}
##                     value              translated count
## 1 Ferreira, Isabel C.F.R. Ferreira, Isabel C.F.R.  1986
## 2             Reis, R. L.             Reis, R. L.  1560
## 3               Adams, T.               Adams, T.  1497
## 4         Sirunyan, A. M.         Sirunyan, A. M.  1450
## 5         Barros, Lillian         Barros, Lillian  1448
## 6            Banerjee, S.            Banerjee, S.  1432
##                                                                                  href
## 1 ?limit=0&type=AllFields&filter%5B%5D=author_facet%3A%22Ferreira%2C+Isabel+C.F.R.%22
## 2             ?limit=0&type=AllFields&filter%5B%5D=author_facet%3A%22Reis%2C+R.+L.%22
## 3               ?limit=0&type=AllFields&filter%5B%5D=author_facet%3A%22Adams%2C+T.%22
## 4         ?limit=0&type=AllFields&filter%5B%5D=author_facet%3A%22Sirunyan%2C+A.+M.%22
## 5         ?limit=0&type=AllFields&filter%5B%5D=author_facet%3A%22Barros%2C+Lillian%22
## 6            ?limit=0&type=AllFields&filter%5B%5D=author_facet%3A%22Banerjee%2C+S.%22
\end{verbatim}

\begin{center}\rule{0.5\linewidth}{0.5pt}\end{center}

Dentro da lista \texttt{status}, é exiba uma mensagem sobre o
\textbf{status} do \texttt{JSON}.

\begin{Shaded}
\begin{Highlighting}[]
\NormalTok{oasisbrDF}\SpecialCharTok{$}\NormalTok{status}
\end{Highlighting}
\end{Shaded}

\begin{verbatim}
## [1] "OK"
\end{verbatim}

\begin{center}\rule{0.5\linewidth}{0.5pt}\end{center}

\hypertarget{r-shiny-app}{%
\subsection{R-Shiny app}\label{r-shiny-app}}

O aplicativo desenvolvido tem o intuito de disponibilizar a visualização
dos indicadores, de maneira interativa. Os scripts estão armazenados no
arquivo \texttt{app.R}. Após finalizado, é necessário disponibilizar a
aplicação.

\begin{center}\rule{0.5\linewidth}{0.5pt}\end{center}

\hypertarget{configurando-o-servidor}{%
\subsection{Configurando o servidor}\label{configurando-o-servidor}}

Algumas etapas são necessárias para configurar o servidor.

\hypertarget{instalar-o-r}{%
\subsubsection{Instalar o R}\label{instalar-o-r}}

Antes de instalar o servidor Shiny, precisamos instalar o R.

\texttt{sudo\ apt-get\ install\ r-base}

\begin{center}\rule{0.5\linewidth}{0.5pt}\end{center}

\hypertarget{instalar-o-rstudio-server}{%
\subsubsection{Instalar o
RStudio-server}\label{instalar-o-rstudio-server}}

A instalação de um servidor Rstudio também é importante para
verificações dos scripts dentro do próprio servidor
\texttt{sudo\ gdebi\ rstudio-server-2021.09.1-372-amd64.deb}

\begin{center}\rule{0.5\linewidth}{0.5pt}\end{center}

\hypertarget{instalar-o-r-shiny-server}{%
\subsubsection{Instalar o R-Shiny
server}\label{instalar-o-r-shiny-server}}

A instalação do Shiny-server é feita com a ferramenta \texttt{GDebi}.
Para sua instalação, basta usar o seguinte comando:

\texttt{sudo\ apt-get\ install\ gdebi-core}

Para fazer o download do Shiny-server, basta usar o seguinte comando
(lembrar de buscar por versão mais atualizada):

\texttt{Wget\ https://download3.rstudio.org/ubuntu-12.04/x86\_64/shiny-server-1.4.2.786-amd64.deb}

Agora, basta usar o \texttt{GDebi} para instalar o arquivo que foi
baixado, utilizando o seguinte comando:

\texttt{sudo\ gdebi\ shiny-server-1.4.2.786-amd64.deb}

\begin{center}\rule{0.5\linewidth}{0.5pt}\end{center}

\hypertarget{instalar-o-pacote-libxml2-dev}{%
\subsubsection{Instalar o pacote
libxml2-dev}\label{instalar-o-pacote-libxml2-dev}}

\texttt{sudo\ apt-get\ update\ -y}
\texttt{sudo\ apt-get\ install\ -y\ libxml2-dev}

\begin{center}\rule{0.5\linewidth}{0.5pt}\end{center}

\hypertarget{instalar-o-pacote-libcurl4-openssl-dev}{%
\subsubsection{Instalar o pacote
libcurl4-openssl-dev}\label{instalar-o-pacote-libcurl4-openssl-dev}}

\texttt{sudo\ apt-get\ update}
\texttt{sudo\ apt-get\ install\ libcurl4-openssl-dev}

\begin{center}\rule{0.5\linewidth}{0.5pt}\end{center}

\hypertarget{acesso-aos-ambientes}{%
\subsection{Acesso aos ambientes}\label{acesso-aos-ambientes}}

\hypertarget{rstudio-server}{%
\subsubsection{Rstudio server}\label{rstudio-server}}

O Rstudio server é disponibilizado na porta \texttt{8787}.

\begin{center}\rule{0.5\linewidth}{0.5pt}\end{center}

\hypertarget{r-shiny-server}{%
\subsubsection{R-Shiny server}\label{r-shiny-server}}

O R-Shiny server é disponibilizado na porta \texttt{3838}.

\begin{center}\rule{0.5\linewidth}{0.5pt}\end{center}

\hypertarget{iniciando-e-parando-o-servidor}{%
\subsubsection{Iniciando e parando o
servidor}\label{iniciando-e-parando-o-servidor}}

Para iniciar ou parar servidor, basta usar os seguintes comando:\\
\texttt{sudo\ systemctl\ start\ shiny-server}
\texttt{sudo\ systemctl\ stop\ shiny-server}

Também é possível reiniciar o servidor, usando:

\texttt{sudo\ systemctl\ restart\ shiny-server}

\hypertarget{pasta-com-aplicativos}{%
\subsubsection{Pasta com aplicativos}\label{pasta-com-aplicativos}}

As aplicações se encontram na pasta \texttt{/srv/shiny-server/}

\begin{center}\rule{0.5\linewidth}{0.5pt}\end{center}

\hypertarget{pasta-com-logs-de-erros}{%
\subsubsection{Pasta com logs de erros}\label{pasta-com-logs-de-erros}}

Todas os logs contendo mensagens de erros e informações sobre problemas
na execução das aplicações, se sencontram na seguinte pasta:
\texttt{/var/log/shiny-server/}

\begin{center}\rule{0.5\linewidth}{0.5pt}\end{center}

\hypertarget{visualizauxe7uxe3o-dos-indicadores}{%
\section{Visualização dos
indicadores}\label{visualizauxe7uxe3o-dos-indicadores}}

\begin{center}\rule{0.5\linewidth}{0.5pt}\end{center}

\hypertarget{autor-author_facet}{%
\subsection{\texorpdfstring{Autor
\texttt{author\_facet}}{Autor author\_facet}}\label{autor-author_facet}}

\begin{Shaded}
\begin{Highlighting}[]
\FunctionTok{library}\NormalTok{(ggplot2)}
\FunctionTok{library}\NormalTok{(scales)}
\FunctionTok{library}\NormalTok{(plotly)}

\NormalTok{author\_facet }\OtherTok{\textless{}{-}}\NormalTok{ oasisbrDF}\SpecialCharTok{$}\NormalTok{facets}\SpecialCharTok{$}\NormalTok{author\_facet}

\DocumentationTok{\#\# Ordena coluna \textquotesingle{}count\textquotesingle{}}
\NormalTok{author\_facet }\OtherTok{\textless{}{-}}\NormalTok{ author\_facet[}\FunctionTok{with}\NormalTok{(author\_facet, }\FunctionTok{order}\NormalTok{(}\SpecialCharTok{{-}}\NormalTok{count)),]}

\DocumentationTok{\#\# Retira registro \textquotesingle{}sem informação\textquotesingle{} da coluna \textquotesingle{}value\textquotesingle{}}
\NormalTok{author\_facet }\OtherTok{\textless{}{-}}\NormalTok{ author\_facet[author\_facet}\SpecialCharTok{$}\NormalTok{value}\SpecialCharTok{!=}\StringTok{\textquotesingle{}sem informação\textquotesingle{}}\NormalTok{,]}

\DocumentationTok{\#\# Seleciona top 10}
\NormalTok{author\_facet }\OtherTok{\textless{}{-}} \FunctionTok{head}\NormalTok{(author\_facet, }\AttributeTok{n=}\DecValTok{10}\NormalTok{)}

\DocumentationTok{\#\# Gráfico de top 10 Autore(a)s}

\NormalTok{authorPlot }\OtherTok{\textless{}{-}} \FunctionTok{ggplot}\NormalTok{(author\_facet) }\SpecialCharTok{+}
  \FunctionTok{aes}\NormalTok{(}\AttributeTok{x =} \FunctionTok{reorder}\NormalTok{(value, count), }\AttributeTok{group =}\NormalTok{ value, }\AttributeTok{weight =}\NormalTok{ count, }
      \AttributeTok{text=}\FunctionTok{paste}\NormalTok{(}\StringTok{"Autor(a):"}\NormalTok{,value,}\StringTok{"\textless{}br\textgreater{}"}\NormalTok{,}\StringTok{"Quantidade"}\NormalTok{,}\FunctionTok{comma}\NormalTok{(count))) }\SpecialCharTok{+}
  \FunctionTok{geom\_bar}\NormalTok{(}\AttributeTok{fill =} \StringTok{"\#112446"}\NormalTok{) }\SpecialCharTok{+}
  \FunctionTok{labs}\NormalTok{(}\AttributeTok{x =} \StringTok{"Nome do autor(a)"}\NormalTok{, }
       \AttributeTok{y =} \StringTok{"Total de documentos"}\NormalTok{, }\AttributeTok{title =} \ConstantTok{NULL}\NormalTok{) }\SpecialCharTok{+}
  \FunctionTok{theme\_minimal}\NormalTok{() }\SpecialCharTok{+}
  \FunctionTok{theme}\NormalTok{(}\AttributeTok{axis.title.x =} \FunctionTok{element\_text}\NormalTok{(}\AttributeTok{size =}\NormalTok{ 14L)) }\SpecialCharTok{+}
  \FunctionTok{coord\_flip}\NormalTok{()}

\FunctionTok{ggplotly}\NormalTok{(authorPlot, }\AttributeTok{tooltip=}\StringTok{"text"}\NormalTok{)}
\end{Highlighting}
\end{Shaded}

\includegraphics{README_files/figure-markdown_github/unnamed-chunk-9-1.png}

\begin{center}\rule{0.5\linewidth}{0.5pt}\end{center}

\hypertarget{aplicauxe7uxe3o-r-shiny---oasisbr}{%
\section{Aplicação R-Shiny -
Oasisbr}\label{aplicauxe7uxe3o-r-shiny---oasisbr}}

\begin{center}\rule{0.5\linewidth}{0.5pt}\end{center}

O principal objetivo da aplicação é exibir visualizações e indicadores
gerais sobre a base de dados do Oasisbr, além da possibilidade de
análises avançadas.

\hypertarget{estrutura-do-app}{%
\subsection{Estrutura do app}\label{estrutura-do-app}}

A aplicação pode ser separada em duas partes:

\texttt{ui}: user side \texttt{server}: server side

Dentro da \texttt{ui} se encontra a interface do usuário para o app, e o
\texttt{server} possui os comandos para gerar os outputs, que são
visualizados no user side, podendo interagir com valores de
\texttt{inputs} fornecidos/selecionados pelo usuário.

A estruturação da \texttt{ui} foi feita utilizando a função
\texttt{tabsetPanel()}, que permite dividir em abas o conteúdo das
\texttt{ui}'s. Neste caso, foram criadas três abas: Indicadores gerais,
Indicadores de evolução e Análises avançadas.

\hypertarget{section-1}{%
\section{}\label{section-1}}

\hypertarget{funuxe7uxe3o-para-download-de-busca-feita-pelo-usuario}{%
\subsection{Função para download de busca feita pelo
usuario}\label{funuxe7uxe3o-para-download-de-busca-feita-pelo-usuario}}

Para a consulta dessas informações, feita via API, foi criada uma função
\texttt{busca\_oasisbr()}, em R, que parametriza a api e faz as
solicitações de acordo com o \texttt{input} do usuário. A busca resulta
em um arquivo no formato \texttt{JSON}.

\begin{Shaded}
\begin{Highlighting}[]
\NormalTok{busca\_oasisbr }\OtherTok{\textless{}{-}} \ControlFlowTok{function}\NormalTok{(}\AttributeTok{url=}\StringTok{"http://localhost/vufind/api/v1/search?"}\NormalTok{,}
\NormalTok{                          lookfor,}
                          \AttributeTok{type=}\StringTok{"AllFields"}\NormalTok{,}
                          \AttributeTok{sort=}\StringTok{"relevance"}\NormalTok{,}
                          \AttributeTok{facet\_parameters=}\StringTok{"\&facet[]=author\_facet\&facet[]=dc.subject.por.fl\_str\_mv\&facet[]=eu\_rights\_str\_mv\&facet[]=dc.publisher.program.fl\_str\_mv\&facet[]=dc.subject.cnpq.fl\_str\_mv\&facet[]=publishDate\&facet[]=language\&facet[]=format\&facet[]=institution\&facet[]=dc.contributor.advisor1.fl\_str\_mv"}\NormalTok{)}
                          
\NormalTok{  \{}
  
\NormalTok{  query }\OtherTok{\textless{}{-}} \FunctionTok{paste}\NormalTok{(url,}\StringTok{"lookfor="}\NormalTok{,}\FunctionTok{URLencode}\NormalTok{(lookfor),}\StringTok{"\&type="}\NormalTok{,type,}\StringTok{"\&sort="}\NormalTok{,sort,facet\_parameters,}\AttributeTok{sep=}\StringTok{""}\NormalTok{)}
  
\NormalTok{  x }\OtherTok{\textless{}{-}} \FunctionTok{fromJSON}\NormalTok{(query)}
  
  \FunctionTok{return}\NormalTok{(x)}
\NormalTok{\}}
\end{Highlighting}
\end{Shaded}

Neste caso, os inputs utilizados são o termo de busca, utilizado na
função como o parâmetro \texttt{lookfor}, e o campo, \texttt{type}, no
qual deve ser feita essa busca. Outros parâmetros também são utilizados,
como \texttt{sort}, que pode ser usado para ordenar o resultado da
busca, e o \texttt{facet\_parameters}, onde são declarados os campos
escolhidos a serem visualizados.

\hypertarget{dataframe-reativo}{%
\subsubsection{Dataframe reativo}\label{dataframe-reativo}}

Ao início da aplicação, é criado um dataframe reativo, utilizando a
função \texttt{reactive()}. O objeto reativo \texttt{oasisbrBuscaUser()}
é criado a partir da função de busca \texttt{busca\_oasisbr()}
previamente mencionada, com o argumento \texttt{lookfor\ =\ ""}, em
branco, por se tratar da busca padrão dentro da base.

\begin{verbatim}
  oasisbrBuscaUser <<- reactive({

    oasisbrDF <- busca_oasisbr(lookfor = "")

    return(oasisbrDF)

  })
\end{verbatim}

Após o carregamento dá aplicação, o usuário pode realizar pesquisas
dentro da base. O input para o texto inserido na barra de busca,
definido na \texttt{ui} da interface, é \texttt{input\$textoBuscaInput}.
O botão de busca, quando ativado, realiza a consulta ao banco utilizando
os inputs \texttt{textoBuscaInput} e \texttt{camposInput}, atualiza o
dataframe reativo \texttt{oasisbrBuscaUser()} e atualiza os outputs dos
gráficos.

\begin{verbatim}
  ## Cria DF reativo para a busca do usuário e atualiza outputs
  
  observeEvent(input$buscarButton,{
    
    print(paste("Iniciado busca para termo:",isolate(input$textoBuscaInput)))
    
    oasisbrBuscaUser <<- reactive({
      
      start <- Sys.time ()
      x <- busca_oasisbr(lookfor = URLencode(isolate(input$textoBuscaInput)),
                    type=isolate(input$camposInput))
      
      tempo_de_busca <<- (Sys.time () - start)
      return(x)
      
    })
    
    mod_graficos_server("graficos")
    
    mod_graficos_evolucao_Server("graficos_evolucao")
    
    mod_analises_avancadas_Server("analises_avancadas")
    
    })
\end{verbatim}

\hypertarget{muxf3dulos}{%
\subsubsection{Módulos}\label{muxf3dulos}}

Para facilitar a organização dos códigos fontes, e também para manter
boas práticas na construção da aplicação, os códigos foram divididos em
\texttt{modulos}. Cada módulo possui sua própria função para \texttt{ui}
e \texttt{server}, o que facilita na implementação de grandes pedaços ou
partes de código em outras aplicações. Os scripts se encontram dentro da
pasta \texttt{R}.

Exemplo de trecho de código do módulo de gráficos da aba de indicadores
gerais: \texttt{R/mod\_graficos.R}

\begin{verbatim}
mod_graficos_UI <- function(id,x) {
  ns <- NS(id)
  

  tagList(
    fluidRow(
             box(
               title = "Instituições com mais documentos", width = 6, solidHeader = TRUE, status = "primary",
               column(12,numericInput(ns("instituicoesTopInput"),"Termos exibidos",min=1, max=25, 10,width="30%")),
               column(12,addSpinner(plotlyOutput(ns("instituicoesPlotlyOutput"),height="300px"),spin="folding-cube",color="green")))
  )
  )
}

mod_graficos_server <- function(id, base) {
  shiny::moduleServer(
    id,
    function(input, output, session) {
      
      output$instituicoesPlotlyOutput <- renderPlotly({render_instituicoesPlot(oasisbrBuscaUser(),input$instituicoesTopInput)})
      
      
    }
  )
}
\end{verbatim}

\hypertarget{funuxe7uxf5es-para-gerar-gruxe1ficos}{%
\subsubsection{Funções para gerar
gráficos}\label{funuxe7uxf5es-para-gerar-gruxe1ficos}}

Foram criadas funções para facilitar a inserção de inputs e visualização
do dados a partir deles. Os scripts se encontram dentro da pasta
\texttt{plots}. A função possui normalmente dois ou mais argumentos,
sendo o primeiro deles o dataframe reativo \texttt{oasisbrBuscaUser()},
e os outros \texttt{inputs} do usuário para filtrar esse objeto. Além
disso, a função realiza manipulações no dataframe, além de criar
validações para o caso de erros ou outputs não desejados. Os gráficos
foram construídos inicialmente utilizando o pacote \texttt{ggplot2}, e,
posteriormente, o pacote \texttt{plotly}, pois este permite maior
flexibilidade e interação do usuário diretamente com o gráfico.

Abaixo, o exemplo da função que retorna a visualização de dados
referentes às instituições presentes na base do Oasisbr.

\begin{verbatim}
  render_instituicoesPlot <- function(x,y) {
  
  ## Validação para busca sem registros
  shiny::validate(need(x$resultCount>0, paste("A sua busca não corresponde a nenhum registro.")))  
  
  instituicoes_facet <- x$facets$institution
  #author_facet
  
  ## Validação para informação vazia.
  shiny::validate(need(is.null(instituicoes_facet)==FALSE, paste("Não existem informações sobre esse(s) registro(s).")))
  
  
  ## Validação para número de termos exibidos
  shiny::validate(need((y>0 & y<=25), paste("O número de termo exibidos precisa estar entre 0 e 25.")))
  
  
  ## Ordena coluna 'count'
  instituicoes_facet <- instituicoes_facet[with(instituicoes_facet, order(-count)),]
  
  ## Retira registro 'sem informação' da coluna 'value'
  instituicoes_facet <- instituicoes_facet[instituicoes_facet$value!='sem informação',]
  
  ## Adiciona % do total
  instituicoes_facet <- instituicoes_facet %>% mutate(pctTotal=count/x$resultCount)
  
  ## Seleciona top 10
  instituicoes_facet <- head(instituicoes_facet, n=y)
  
  instituicoes_facet$color <- "#76B865"
  
  
  ## Gráfico de top 10 Autore(a)s
  
  instituicoesPlot <- ggplot(instituicoes_facet) +
    aes(x = reorder(value, count), group = value, weight = count, 
        text=paste('<b style="font-family: Lato !important; align=left; font-size:14px; font-weight:400; color:gray">Instituição:</b>',
                   '<b style="font-family: Lato !important; align=left; font-size:16px; font-weight:600 color: black">',value,"</b>",
                   "<br><br>",
                   '<b style="font-family: Lato !important; align=left; font-size:14px font-weight:400; color:gray">Total de documentos:</b>',
                   '<b style="font-family: Lato !important; align=left; font-size:16px; font-weight:600 color: black">',comma(count),"</b>",
                   "<br><br>")
    ) +
    geom_bar(fill = "#76B865") +
    
    scale_y_continuous(labels = scales::comma)+
    labs(x = "<b style='color:gray'>Instituição</b><br><br><b style='color:white'>.", 
         y = "<b style='color:gray; font-size:14px'>Total de documentos", title = NULL) +
    
    theme_minimal() +
    theme(axis.title.x = element_text(size = 14L)) +
    coord_flip()
  
  instituicoesPlot <- ggplotly(instituicoesPlot, tooltip="text")
  
  instituicoesPlot %>%
    
    layout(font=t, 
           margin = list(l=50,b = 55),
           hoverlabel=list(bgcolor="white")
    ) %>% config(displayModeBar = F) 
  
  
  
  
}
\end{verbatim}

\hypertarget{outputs}{%
\subsubsection{Outputs}\label{outputs}}

Os outputs são definidos no \texttt{server} da aplicação. Como os
objetos gráficos criados foram feitos utilizando o pacote
\texttt{plotly}, precisamos utilizar a função
\texttt{renderPlotly(\{\})} para renderizá-los. São declarados dentro do
output a função que renderiza o gráfico em questão (neste caso,
\texttt{render\_instituicoesPlot()}), com o dataframe reativo
\texttt{oasisbrBuscaUser()} e um input
\texttt{input\$instituicoesTopInput} herdado da \texttt{ui}, servindo
como filtro para a visualização.

\begin{verbatim}
output$instituicoesPlotlyOutput <- renderPlotly({

render_instituicoesPlot(oasisbrBuscaUser(),input$instituicoesTopInput)

})
\end{verbatim}

\hypertarget{indicadores-gerais}{%
\subsubsection{Indicadores Gerais}\label{indicadores-gerais}}

Dentro da aba de indicadores gerais são exibidos nove gráficos,
referentes aos campos que são recuperados pela consulta ao banco.

Ao lado esquerdo da imagem, a página inicial, e ao lado direito, os
indicadores atualizados para a busca pelo termo ``biotecnologia'' como
exemplo. \#

\hypertarget{indicadores-de-evoluuxe7uxe3o-1}{%
\subsubsection{Indicadores de
evolução}\label{indicadores-de-evoluuxe7uxe3o-1}}

Os indicadores de evolução são coletados pela api disponível em:
\texttt{https://api-oasisbr.ibict.br/api/v1/evolution-indicators?init=10/10/2017\&end=10/10/2021}.

\hypertarget{section-2}{%
\section{}\label{section-2}}

\hypertarget{anuxe1lises-avanuxe7adas}{%
\subsubsection{Análises avançadas}\label{anuxe1lises-avanuxe7adas}}

Abaixo, uma das análises avançadas disponíveis na máquina de testes, o
heatmap da quantidade de documentos por instituições e por ano de
publicação.

\hypertarget{section-3}{%
\section{}\label{section-3}}

Outras análises avançadas estão em andamento, como a análise de redes de
colaboraçãoo, visualização em mapas e disponibilização de
\texttt{pivotTable} para análise livre por parte do usuário.

\hypertarget{customizauxe7uxe3o-css}{%
\subsubsection{Customização CSS}\label{customizauxe7uxe3o-css}}

A aplicação também pode ser customizada, utilizando CSS. Abaixo é
exibido o código utilizado para a padronização de fontes e cores do
aplicativo, além de ajustes finos de layout.

\end{document}
